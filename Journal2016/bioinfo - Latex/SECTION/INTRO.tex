Biomedical research increasingly depends on large databases, such as UniProt  \citep{UniProtConsortium2014} or Ensembl \citep{Cunningham2014}. The formulation of database queries and the correct interpretation of query results requires knowledge of the database structure as well as implicit background knowledge about the way data is produced and recorded. 
Such background knowledge may vary between users, thus biasing uniform data interpretation. 

There have been several attempts to enhance data retrieval by formal ontologies like ontology-based data access (OBDA) \citep{Poggi2008} or the combination of ontologies with machine learning \citep{Lehmann2009}. However, such approaches require in-depth manual interpretation of the finally retrieved content; and the representation of data entities  as individuals (ABox elements in description logics) results in high processing cost \citep{Hustadt2005}.

Accordingly, the current situation regarding the (semi-)automated support for database content retrieval and interpretation is characterized by a concurrent and continuous evolution of high-quality structured knowledge resources, whereas less progress can be seen regarding their interoperability and ontological underpinning. Superficial use of ontologies - just as vocabularies - restricts their applicability. We hypothesise that with a formally explicit view on biomedical data as provided by principled ontologies, users are better served to integrate, retrieve, validate and interpret these data, due to automated classification and consistency checking provided by Description Logics (DL) representation and reasoning \citep{Baader2007g}. 
% classifiers, like Fact++ \citep{Tsarkov2006}.

We investigate this hypothesis by proposing a model for the seamless integration of the content of biological databases and ontologies, guided by formal-ontological principles \citep{Smith2007}, which enforce univocal interpretation of database content in order to improve (semi-)automated data interpretation and understanding. Ontology-level content in identified in databases, and axiomatized under an upper level ontology. Thus it delivers a homogeneous representation of data, linked to (modules of) existing ontologies.
The proposed approach might be applied to enhance database curation, as new statements extracted from publications and recorded in databases can be tested for adequacy. Supported by description logics (DL) based reasoning, it might enable validation using DL Query as an expressive and simple query language based on DL syntax and semantics. 

We will substantiate this claim by investigating a subset of biomedical ontologies and databases. We (i) propose an ontology-based framework that makes explicit both database content and the domain entities denoted thereby; (ii) relate this framework to current workflows in which life science data and knowledge are acquired and processed; (iii) implement an example ontology from real data as an exemplar for database-ontology integration; (iv) validate it by demonstrating how querying is done by using DL Query; and (vi) experimentally assess correctness and scalability.

The biological use case is addressed by Competency Questions (CQs) \citep{Gruninger1994} as DL queries on the metabolism of small molecules in model organisms, encompassing physiological processes as well as phenotypical changes in dysfunctional  metabolism. The examples make use of UniProt, NCBI Taxonomy \citep{NCBI2015}, Ensembl\citep{Cunningham2014}, SNOMED CT \citep{IHTSDO2015}, GO \citep{Gene2014a}, ChEBI \citep{Hastings2013} and PR \citep{Natale2014}, organized under the upper domain ontology BioTopLite (BTL2) \citep{Schulz2012}. Our approach capitalizes on previous work about the formalization of tabular representations in scientific literature \citep{Santana2011b} and models of structured clinical information \citep{Martinez-Costa2015}.


%Hypotheses and findings are generated in biomedical research based on gathering data from experiments, publications, and databases like UniProt \citep{UniProtConsortium2014} or Ensembl \citep{Cunningham2014}. In the biomedical field, the exploration of this contents is usually performed manually or (partly) supported by retrieval tools, e.g. by STRING \citep{Szklarczyk2014}, or BLAST \citep{Altschup1990}. Some works highlight the need and importance of database interpretation for the biomedical domain \citep{Carnielli2015,Laukens2015}.

%Moreover, the interpretation of these results may be biased by the researchers' capabilities, by the sheer size and heterogeneity of the sources, or by technical limitations \citep{Triplet2011,Laukens2015}. It may result in ambiguous interpretation of data, the non-observance of records that may be crucial for a given problem, or by the user inability to query and filter a dataset, or the elevated processing costs related to massive datasets.
%
%On the other hand, others tools like ontology-based data access (OBDA) \citep{Poggi2008} -related applications enhance data retrieval by ontologies, which serve as a query vocabulary -- e.g. within SPARQL \citep{Harris2013} endpoints. By using, one gains the ability to integrate heterogeneous and massive data sources. Other tools rely on machine learning to interpret databases according to an ontological background \citep{Lehmann2009}. The retrieval of OBDA-based approaches, like SPARQL endpoints, fairly supports reasoning that goes beyond what is available in current relational queries \citep{Angles2008a}. 
%
%However, such approaches are limited by the need of user intervention, e.g. the manual interpretation of the finally retrieved content, and the ontology population with data. Here, the interpretation problem is still present, as well as the raise of processing costs. OBDA and machine learning approaches frequently rely on the representation of data entities as individuals (ABox elements in DL jargon), which results in high processing cost \citep{Hustadt2005}, with or without computer-assisted reasoning.
%
%Accordingly, the current situation regarding the (semi-)automated support for database content retrieval and interpretation is characterized by a concurrent and continuous evolution of high-quality structured knowledge resources, whereas less progress can be seen regarding their usage, interoperability and ontological grounding. Thus, the interpretation of results is frequently left to the users, and influenced by implicit background assumptions that may vary between users. 
%
%Apart from that, the usage of ontologies only to enable data integration and retrieval restricts its applicability and justification. With a formalized view of biomedical data provided by ontologies, we are able to integrate, retrieve, validate and derive assumptions from it, mainly due to automated classification and consistency checking delivered by Description Logics (DL) \citep{Baader2007g} classifiers, like HermiT \citep{Glimm2014}.
%
%We address this shortcoming by advocating a seamless integration of database and ontology content, underpinned by formal-ontological principles \citep{Smith2007}, which enforce an univocal interpretation of the database content. In other words, we emphasize ontological grounding of databases as a mean enable (semi-)automated data interpretation and understanding.
%Ontological grounding means to identify ontology-level content in databases, and to axiomatize it under an upper level ontology. It delivers a homogeneous representation of data, linked to (parts of) existing ontologies. 
%
%Ontological grounding can be applied to enhance database curation, as new statements extracted from recent publications and recorded in databases can be ontologically grounded and tested for adequacy inside the biological domain.  As this process is supported by automated reasoning, it enables a rigorous validation supported by an expressive query language (DL Query). 
%
%A DL Query uses the semantics and reasoning procedures of DL, which allow querying across several databases at the same time, thus decreasing the costs of database integration.  This "ontological grounding" is based on previous work on the formalization of tabular representations in scientific literature \citep{Santana2011b} and models of structured clinical information \citep{Martinez-Costa2015}.
%
%In this sense, we hypothesise and demonstrate that this enables more powerful queries, supported by machine reasoning.  To this end, we will (i) analyse a subset of biomedical ontologies and databases; (ii) propose an ontology-based framework that makes explicit both database content and the domain entities denoted thereby; (iii) relate this solution to current workflows in which life science data and knowledge are acquired and processed; (iv) implement an example ontology from real data as an exemplar for data integration across ontologies and databases; (v) validate this example by demonstrating how querying becomes simpler and more user-friendly; (vi) perform an experiment to assess the scalability of the approach. 
%
%The biological use case is addressed Competency Questions (CQs) \citep{Gruninger1994} as DL queries. They range from how data related to a specific metabolism overlap in biological databases for certain model organisms, to phenotypes from dysfunctional metabolism. Our examples make use of UniProt, NCBI Taxonomy \citep{NCBI2015}, Ensembl, SNOMED CT \citep{Donnelly2006} , GO \citep{Gene2014a}, ChEBI \citep{Hastings2013} and PR \citep{Natale2014}, organized under the upper domain ontology BioTopLite (BTL2) \citep{Schulz2013}. 

